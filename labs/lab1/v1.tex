\documentclass[]{article}
\usepackage{lmodern}
\usepackage{amssymb,amsmath}
\usepackage{ifxetex,ifluatex}
\usepackage{fixltx2e} % provides \textsubscript
\ifnum 0\ifxetex 1\fi\ifluatex 1\fi=0 % if pdftex
  \usepackage[T1]{fontenc}
  \usepackage[utf8]{inputenc}
\else % if luatex or xelatex
  \ifxetex
    \usepackage{mathspec}
  \else
    \usepackage{fontspec}
  \fi
  \defaultfontfeatures{Ligatures=TeX,Scale=MatchLowercase}
\fi
% use upquote if available, for straight quotes in verbatim environments
\IfFileExists{upquote.sty}{\usepackage{upquote}}{}
% use microtype if available
\IfFileExists{microtype.sty}{%
\usepackage{microtype}
\UseMicrotypeSet[protrusion]{basicmath} % disable protrusion for tt fonts
}{}
\usepackage[margin=1in]{geometry}
\usepackage{hyperref}
\PassOptionsToPackage{usenames,dvipsnames}{color} % color is loaded by hyperref
\hypersetup{unicode=true,
            pdftitle={Lab Report no 1 for ST344},
            pdfauthor={1632905},
            colorlinks=true,
            linkcolor=Maroon,
            citecolor=Blue,
            urlcolor=blue,
            breaklinks=true}
\urlstyle{same}  % don't use monospace font for urls
\usepackage{color}
\usepackage{fancyvrb}
\newcommand{\VerbBar}{|}
\newcommand{\VERB}{\Verb[commandchars=\\\{\}]}
\DefineVerbatimEnvironment{Highlighting}{Verbatim}{commandchars=\\\{\}}
% Add ',fontsize=\small' for more characters per line
\usepackage{framed}
\definecolor{shadecolor}{RGB}{248,248,248}
\newenvironment{Shaded}{\begin{snugshade}}{\end{snugshade}}
\newcommand{\AlertTok}[1]{\textcolor[rgb]{0.94,0.16,0.16}{#1}}
\newcommand{\AnnotationTok}[1]{\textcolor[rgb]{0.56,0.35,0.01}{\textbf{\textit{#1}}}}
\newcommand{\AttributeTok}[1]{\textcolor[rgb]{0.77,0.63,0.00}{#1}}
\newcommand{\BaseNTok}[1]{\textcolor[rgb]{0.00,0.00,0.81}{#1}}
\newcommand{\BuiltInTok}[1]{#1}
\newcommand{\CharTok}[1]{\textcolor[rgb]{0.31,0.60,0.02}{#1}}
\newcommand{\CommentTok}[1]{\textcolor[rgb]{0.56,0.35,0.01}{\textit{#1}}}
\newcommand{\CommentVarTok}[1]{\textcolor[rgb]{0.56,0.35,0.01}{\textbf{\textit{#1}}}}
\newcommand{\ConstantTok}[1]{\textcolor[rgb]{0.00,0.00,0.00}{#1}}
\newcommand{\ControlFlowTok}[1]{\textcolor[rgb]{0.13,0.29,0.53}{\textbf{#1}}}
\newcommand{\DataTypeTok}[1]{\textcolor[rgb]{0.13,0.29,0.53}{#1}}
\newcommand{\DecValTok}[1]{\textcolor[rgb]{0.00,0.00,0.81}{#1}}
\newcommand{\DocumentationTok}[1]{\textcolor[rgb]{0.56,0.35,0.01}{\textbf{\textit{#1}}}}
\newcommand{\ErrorTok}[1]{\textcolor[rgb]{0.64,0.00,0.00}{\textbf{#1}}}
\newcommand{\ExtensionTok}[1]{#1}
\newcommand{\FloatTok}[1]{\textcolor[rgb]{0.00,0.00,0.81}{#1}}
\newcommand{\FunctionTok}[1]{\textcolor[rgb]{0.00,0.00,0.00}{#1}}
\newcommand{\ImportTok}[1]{#1}
\newcommand{\InformationTok}[1]{\textcolor[rgb]{0.56,0.35,0.01}{\textbf{\textit{#1}}}}
\newcommand{\KeywordTok}[1]{\textcolor[rgb]{0.13,0.29,0.53}{\textbf{#1}}}
\newcommand{\NormalTok}[1]{#1}
\newcommand{\OperatorTok}[1]{\textcolor[rgb]{0.81,0.36,0.00}{\textbf{#1}}}
\newcommand{\OtherTok}[1]{\textcolor[rgb]{0.56,0.35,0.01}{#1}}
\newcommand{\PreprocessorTok}[1]{\textcolor[rgb]{0.56,0.35,0.01}{\textit{#1}}}
\newcommand{\RegionMarkerTok}[1]{#1}
\newcommand{\SpecialCharTok}[1]{\textcolor[rgb]{0.00,0.00,0.00}{#1}}
\newcommand{\SpecialStringTok}[1]{\textcolor[rgb]{0.31,0.60,0.02}{#1}}
\newcommand{\StringTok}[1]{\textcolor[rgb]{0.31,0.60,0.02}{#1}}
\newcommand{\VariableTok}[1]{\textcolor[rgb]{0.00,0.00,0.00}{#1}}
\newcommand{\VerbatimStringTok}[1]{\textcolor[rgb]{0.31,0.60,0.02}{#1}}
\newcommand{\WarningTok}[1]{\textcolor[rgb]{0.56,0.35,0.01}{\textbf{\textit{#1}}}}
\usepackage{graphicx,grffile}
\makeatletter
\def\maxwidth{\ifdim\Gin@nat@width>\linewidth\linewidth\else\Gin@nat@width\fi}
\def\maxheight{\ifdim\Gin@nat@height>\textheight\textheight\else\Gin@nat@height\fi}
\makeatother
% Scale images if necessary, so that they will not overflow the page
% margins by default, and it is still possible to overwrite the defaults
% using explicit options in \includegraphics[width, height, ...]{}
\setkeys{Gin}{width=\maxwidth,height=\maxheight,keepaspectratio}
\IfFileExists{parskip.sty}{%
\usepackage{parskip}
}{% else
\setlength{\parindent}{0pt}
\setlength{\parskip}{6pt plus 2pt minus 1pt}
}
\setlength{\emergencystretch}{3em}  % prevent overfull lines
\providecommand{\tightlist}{%
  \setlength{\itemsep}{0pt}\setlength{\parskip}{0pt}}
\setcounter{secnumdepth}{0}
% Redefines (sub)paragraphs to behave more like sections
\ifx\paragraph\undefined\else
\let\oldparagraph\paragraph
\renewcommand{\paragraph}[1]{\oldparagraph{#1}\mbox{}}
\fi
\ifx\subparagraph\undefined\else
\let\oldsubparagraph\subparagraph
\renewcommand{\subparagraph}[1]{\oldsubparagraph{#1}\mbox{}}
\fi

%%% Use protect on footnotes to avoid problems with footnotes in titles
\let\rmarkdownfootnote\footnote%
\def\footnote{\protect\rmarkdownfootnote}

%%% Change title format to be more compact
\usepackage{titling}

% Create subtitle command for use in maketitle
\providecommand{\subtitle}[1]{
  \posttitle{
    \begin{center}\large#1\end{center}
    }
}

\setlength{\droptitle}{-2em}

  \title{Lab Report no 1 for ST344}
    \pretitle{\vspace{\droptitle}\centering\huge}
  \posttitle{\par}
    \author{1632905}
    \preauthor{\centering\large\emph}
  \postauthor{\par}
    \date{}
    \predate{}\postdate{}
  

\begin{document}
\maketitle

\hypertarget{background}{%
\section{1. Background}\label{background}}

The task we were given is to examine the relationship between
self-perceived attractiveness and the the variablewhich measureshow much
one person believesthat their data wants to meet with them again, to
which we will refer as \emph{date likelihood}.

My research will be based on the results of the experiment which was
conducted by researchers for Columbia, Harvard and Stanford
Universities. The raw data can be found on
\href{http://www.stat.columbia.edu/~gelman/arm/examples/speed.dating/}{this
website}.

The researchers Fisman, Iyengar, Kamenica and Simonson aimed to find how
people choose their dating and/or marriage partners. Afterthoroough
consideration, they decided to pursue speed dating experiment to gather
necessary data for their research. Before the events, the potential
participants were asked to fill out a online survey which asked them to
rate themselves on the following attributes: : attractiveness,
sincerity, intelligence, fun and ambition. They also provided various
demographical information, such as their ethnicity, their field of
study, the name of the university at which they studied or had been
undergraduates, and their home location (ZIP code). As the sutdy focused
only on heterosexual couples, during each event the female participant
had a 4-minute conversation with a maleparticipant, after which they
filled out a post-meeting score sheet and proceeded to their next
partners. Therefore, during on event, each participant took part in more
than one date.

\hypertarget{introduction}{%
\section{2. Introduction}\label{introduction}}

We stard our work with uploading the necessary data.

\begin{Shaded}
\begin{Highlighting}[]
\NormalTok{SpeedRawData <-}\StringTok{ }\KeywordTok{read.csv}\NormalTok{(}\StringTok{"SpeedDatingRawData.csv"}\NormalTok{)}
\NormalTok{SpeedData <-}\StringTok{ }\KeywordTok{read.csv}\NormalTok{(}\StringTok{"SpeedData.csv"}\NormalTok{)}
\end{Highlighting}
\end{Shaded}

One of our main variables will be self-perceived attractiveness as
mentioned above. It will be represented by variable \emph{attr3\_1}
which has a following description in the survey:\\
\emph{``Please rate your opinion of your own attributes, on a scale of
1-10 (be honest!): Attractive''}\\
We purposefully chose not to take into account variables attr3\_2,
atter3\_3 and attr3\_4. Although they also measured self-attractiveness,
they were colleted some time after the specific event and may be biased
because of later dates or other events which can perturb our
measurements.

As the variable which measures \emph{date likelihood} we will use
\emph{prob} which is the anser to the question: \emph{How probable do
you think it is that this person will say `yes' for you? (1=not
probable, 10=extremely probable)}

\hypertarget{data-preparation}{%
\section{3. Data preparation}\label{data-preparation}}

We must point that we will use onle each participant's first speed data,
as the next ones occur in very short periods of time and could be not
mutually independent. As part of data cleaning, we want also to erase
rows without values in them. In order to perform those tasks, we
prepared a function:

\begin{Shaded}
\begin{Highlighting}[]
\NormalTok{DataCleaning<-}\ControlFlowTok{function}\NormalTok{(RawData, }\DataTypeTok{waves=}\KeywordTok{c}\NormalTok{(}\DecValTok{1}\OperatorTok{:}\DecValTok{21}\NormalTok{), }\DataTypeTok{attribute1=}\StringTok{"prob"}\NormalTok{, }\DataTypeTok{attribute2=}\StringTok{"attr3_1"}\NormalTok{)}
\NormalTok{\{}
\NormalTok{part <-}\StringTok{ }\NormalTok{(SpeedRawData}\OperatorTok{$}\NormalTok{partner }\OperatorTok{==}\StringTok{ }\DecValTok{1}\NormalTok{)}
\NormalTok{SpeedData <-}\StringTok{ }\NormalTok{RawData[part, }\KeywordTok{c}\NormalTok{(}\StringTok{"iid"}\NormalTok{, }\StringTok{"gender"}\NormalTok{, }\StringTok{"prob"}\NormalTok{, }\StringTok{"attr3_1"}\NormalTok{)]}
\NormalTok{SpeedData<-SpeedData[}\KeywordTok{complete.cases}\NormalTok{(SpeedData),]}
\KeywordTok{return}\NormalTok{(SpeedData)}
\NormalTok{\}}
\end{Highlighting}
\end{Shaded}

\hypertarget{analysis}{%
\section{4. Analysis}\label{analysis}}

For the purpose of this analysis, we will construct function which will
create boxplots for our data.\\
We will plot data for males and females separately to examine if there
are differences between them. In our dataset males are denoted as
\emph{1} and females as \emph{0}.

\begin{Shaded}
\begin{Highlighting}[]
\NormalTok{PlotData <-}\StringTok{ }\ControlFlowTok{function}\NormalTok{(inputfile, }\DataTypeTok{waves =} \KeywordTok{c}\NormalTok{(}\DecValTok{1}\OperatorTok{:}\DecValTok{21}\NormalTok{),}
                     \DataTypeTok{attribute1 =} \StringTok{"prob"}\NormalTok{, }\DataTypeTok{attribute2 =} \StringTok{"attr3_1"}\NormalTok{)}
\NormalTok{\{}
\NormalTok{  Data <-}\StringTok{ }\KeywordTok{DataCleaning}\NormalTok{(inputfile)}
  \KeywordTok{par}\NormalTok{(}\DataTypeTok{mfrow=}\KeywordTok{c}\NormalTok{(}\DecValTok{1}\NormalTok{,}\DecValTok{2}\NormalTok{))}
  \KeywordTok{boxplot}\NormalTok{(Data[,}\DecValTok{3}\NormalTok{] }\OperatorTok{~}\StringTok{ }\NormalTok{Data[,}\DecValTok{4}\NormalTok{], }\DataTypeTok{data =}\NormalTok{ Data, }\DataTypeTok{subset =}\NormalTok{ (gender}\OperatorTok{==}\DecValTok{1}\NormalTok{),}
          \DataTypeTok{main =} \StringTok{"Relationship for males"}\NormalTok{, }\DataTypeTok{col =}\NormalTok{ (}\KeywordTok{c}\NormalTok{(}\StringTok{"lightblue"}\NormalTok{, }\StringTok{"darkblue"}\NormalTok{)),}
          \DataTypeTok{xlab =} \StringTok{"Self-perceived attractiveness"}\NormalTok{, }\DataTypeTok{ylab =} \StringTok{"Date Likelihood"}\NormalTok{)}
  \KeywordTok{boxplot}\NormalTok{(Data[,}\DecValTok{3}\NormalTok{] }\OperatorTok{~}\StringTok{ }\NormalTok{Data[,}\DecValTok{4}\NormalTok{], }\DataTypeTok{data =}\NormalTok{ Data, }\DataTypeTok{subset =}\NormalTok{ (gender}\OperatorTok{==}\DecValTok{0}\NormalTok{),}
          \DataTypeTok{main =} \StringTok{"Relationship for females"}\NormalTok{, }\DataTypeTok{col =}\NormalTok{ (}\KeywordTok{c}\NormalTok{(}\StringTok{"lightblue"}\NormalTok{, }\StringTok{"darkblue"}\NormalTok{)),}
          \DataTypeTok{xlab =} \StringTok{"Self-perceived attractiveness"}\NormalTok{, }\DataTypeTok{ylab =} \StringTok{"Date Likelihood"}\NormalTok{)}
\NormalTok{\}}
\end{Highlighting}
\end{Shaded}

We use this function to create box plots for our data.

\begin{Shaded}
\begin{Highlighting}[]
\KeywordTok{PlotData}\NormalTok{(SpeedRawData)}
\end{Highlighting}
\end{Shaded}

\includegraphics{v1_files/figure-latex/chunk4-1.pdf}

From the first boxplot, we see that there is the strong positive
correlation between self-perceived attractiveness and data likelihood
for males. The mean of data likelihood is higher for males with higher
self-perceived attractiveness.\\
However, this relationsship is not that clear for females. The trend is
mostly positive but the relationship is much less linear in this case.

Furthermore, we see from the plots that some people with very high
self-perceived attractiveness rate their dating chances very low as the
plot shows that there are people who rated their attractiveness on 8-10
and still thinks that their Date Likelihood is in the range 0-2. Also,
we can find out that the median of Date Likelihood for very attractive
females with self-perceived attractiveness of 8 or 9, is only 6.

\hypertarget{conclusions}{%
\section{6. Conclusions}\label{conclusions}}

We found out that typically people with hiigher self-perceived
attractiveness have stronger beliefs that the other person will date
them again. Thir correlation is stronger among males than females. In
the second case, the relationsheep is more flat, as even females with
very high self-perceived attractiveness are not sure if other person is
interested in them.

\hypertarget{references}{%
\section{7. References}\label{references}}

Fisman, R., S. S. Iyengar, E. Kamenica, and I. Simonson. 2006. ``Gender
Differences in Mate Selection: Evidence from a Speed Dating
Experiment.'' The Quarterly Journal of Economics 121 (2). MIT Press:
673--97.


\end{document}
